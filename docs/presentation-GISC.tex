\documentclass[10pt]{beamer}

\usepackage[T1]{fontenc}
\usepackage[utf8]{inputenc}

\usepackage{lmodern}

\usepackage{mathtools,amssymb}

\usepackage{graphicx}
\usepackage{xcolor}
\usepackage{tikz}

\usepackage{booktabs}

\usepackage{appendixnumberbeamer}

\definecolor{red_UCM}{HTML}{b01131}
\usetheme{Berlin}
\useoutertheme{infolines}
\useinnertheme{rectangles}
\usecolortheme[named=red_UCM]{structure}
\setbeamercolor{alerted text}{fg=red_UCM}

\usetikzlibrary{calc, positioning}
\AtBeginSection[]{\begin{frame}\tableofcontents[currentsection]\end{frame}}
\setbeameroption{show notes}
\setbeamertemplate{note page}{\vskip 0.5em \insertnote}
\addtobeamertemplate{note page}{}{\thispdfpagelabel{notes:\insertframenumber}}
\setbeamertemplate{navigation symbols}{}
\usebackgroundtemplate{\tikz[overlay,remember picture]\node[opacity=0.03125]at (current page.center){\includegraphics[scale=0.5]{images/UCM-logo.png}};}
\renewcommand\appendixname{Appendix}

\title[XXII GISC Workshop]{Evolving Safely: Adaptation and Phenotypic Diversity under Environmental and Demographic Constraints}
\author[]
{
  \texorpdfstring{{\large Mendívil Carboni, Marco\inst{1}}\\\vspace{1ex} Mazo Torres, Juan José\inst{1} (dir.) \and Dinis Vizcaíno, Luis Ignacio\inst{1} (dir.)}{Marco Mendívil Carboni}
} 
\institute[Universidad Complutense de Madrid]
{
  \inst{1}
  Grupo Interdisciplinar de Sistemas Complejos (GISC) and Departamento de Estructura de la Materia, Física Térmica y Electrónica, Universidad Complutense de Madrid
}

\begin{document}

\frame{\titlepage}

\begin{frame}
  \tableofcontents
\end{frame}

\section{Introduction and objectives}

\begin{frame}
  \begin{minipage}{0.62\textwidth}
    \begin{block}{Motivation}
      \begin{itemize}
        \item \alert{Adaptation} in fluctuating \alert{environments} is not fully understood yet.
        \item Especially under \alert{demographic constraints}.
        \item Paradigmatic setting to study the effects of \alert{fluctuations in biological processes}.
      \end{itemize}
    \end{block}
    \vfill
  \end{minipage}
  \hfill
  \begin{minipage}{0.34\textwidth}
    \begin{figure}
      \centering
      \frame{\includegraphics[scale=0.12]{images/changing-environment.jpg}}
    \end{figure}
  \end{minipage}

  \begin{block}{Objectives}
    \begin{itemize}
      \item Develop a minimal mathematical \alert{model} to study this problem.
      \item Quantify how the \alert{evolutionary outcome} depends on these constraints.
      \item Identify the main mechanisms shaping these results and seek ways to \alert{approximate} them.
    \end{itemize}
  \end{block}

  \note[item]{Adaptation in fluctuating environments is a \alert{central problem} in \alert{evolutionary biology}.}
  \note[item]{Natural populations experience fluctuating rather than constant environments.}
  \note[item]{We will use a minimal mathematical model, \alert{not a detailed biological one}.}
  \note[item]{To investigate this, we will use \alert{numerical simulations}.}
  \note[item]{The main objective is to understand \alert{how populations adapt in fluctuating environments under demographic constraints}.}
  \note[item]{A key question is whether these outcomes can be described using an \alert{effective notion of fitness}.}
\end{frame}

\section{The model: a Markov process}

\begin{frame}
  \begin{block}{Model definition}
    \alert{Continuous-time Markov process}:
    \begin{itemize}
      \item \alert{Environment}: $e\in\{0,\dots,n_{\text{env}}-1\}$, evolves with rates \alert{$\omega_t(e,e')$}.
      \item \alert{Agents}: population of size $N\le N_i$, each agent characterized by:
            \begin{itemize}
              \item \alert{phenotype}: $\phi\in\{0,\dots,n_{\text{phe}}-1\}$.
              \item \alert{phenotypic strategy}: $s(\phi)$ (probability distribution).
            \end{itemize}
      \item \alert{Dynamics}: birth--death process with environment- and phenotype-dependent rates \alert{$\omega_b(e,\phi)$} and \alert{$\omega_d(e,\phi)$}.
      \item \alert{Evolution}: offsprings inherit the parent's strategy; mutations occur with probability \alert{$p_{\text{mut}}$}.
      \item \alert{Constraint}: population size \alert{capped at $N_i$}.
    \end{itemize}
  \end{block}

  \note[item]{The model is a modified birth-and-death Markov process describing a population evolving in a fluctuating environment.}
  \note[item]{The population starts with an initial number of agents and evolves in continuous time.}
  \note[item]{The environment is a discrete variable that can take one of a finite number of states and changes according to a continuous-time Markov chain with given transition rates.}
  \note[item]{Each agent is characterized by a phenotype and a phenotypic strategy.}
  \note[item]{The phenotype takes discrete values, while the phenotypic strategy is a probability distribution over the possible phenotypes.}
  \note[item]{When an agent reproduces, the offspring's phenotype is drawn from the parent's phenotypic strategy.}
  \note[item]{Offspring inherit the same phenotypic strategy as their parent, allowing natural selection to act on strategies.}
  \note[item]{To allow evolution, mutations are introduced: with a small probability at duplication, an agent's strategy mutates to a new random strategy.}
  \note[item]{Although rare, these mutations enable exploration of the space of possible strategies during simulations.}
  \note[item]{The total population size is capped at its initial value, introducing a demographic constraint.}
\end{frame}

\begin{frame}
  \begin{block}{Parameters}
    \begin{table}
      \centering
      \begin{tabular}{c c c}
        \toprule
        symbol             & name                         & value                                                        \\
        \midrule
        $\omega_t(e,e')$   & environment transition rates & $\begin{psmallmatrix}-1.0&+1.0\\+1.0&-1.0\end{psmallmatrix}$ \\
        \addlinespace
        $\omega_b(e,\phi)$ & birth rates                  & $\begin{psmallmatrix}1.0&0.2\\0.0&0.0\end{psmallmatrix}$     \\
        \addlinespace
        $\omega_d(e,\phi)$ & death rates                  & $\begin{psmallmatrix}0.0&0.0\\1.0&0.1\end{psmallmatrix}$     \\
        \addlinespace
        $p_{\text{mut}}$   & mutation probability         & $0.001$                                                      \\
        \addlinespace
        $N_i$              & initial number of agents     & $100$                                                        \\
        \bottomrule
      \end{tabular}
    \end{table}
  \end{block}

  \note[item]{The analysis is restricted to the case of two environments and two phenotypes.}
  \note[item]{The stochastic dynamics are simulated using the Gillespie algorithm.}
  \note[item]{Measured observables include time, population size, and the population growth rate defined as $\Delta N/(N\Delta t)$.}
  \note[item]{The number of extinctions is recorded to compute extinction rates.}
  \note[item]{We compute the average phenotypic strategy across the population and over time.}
  \note[item]{We also compute the standard deviation of the phenotypic strategy across agents.}
  \note[item]{Histograms are constructed for phenotypic strategies and for phenotype proportions in the population.}
  \note[item]{Because strategies and phenotypes are two-dimensional and normalized, only the first component needs to be tracked.}
\end{frame}

\begin{frame}
  \note[item]{environmental transition rates are equal and set to one, making both environments equally likely.}
  \note[item]{The mutation probability is set to $10^{-3}$, and the maximum population size is $100$.}
  \note[item]{One phenotype grows well in a good environment but dies quickly in a bad one.}
  \note[item]{The second phenotype grows more slowly in the good environment but survives much better in the bad environment.}
  \note[item]{Three types of simulations are performed: fixed, evolutive, and random.}
  \note[item]{In fixed simulations, all agents share a prescribed, non-evolving strategy and mutations are disabled.}
  \note[item]{In evolutive simulations, the population starts monomorphic, and strategies evolve through mutation and selection.}
  \note[item]{In random simulations, agents start with independent random strategies, which then evolve.}
  \note[item]{These cases are compared to identify evolutionary attractors and the strategies favored by selection.}
  \note[item]{The extended parameter set is identical to the asymmetric case but uses a larger population size of $1000$ agents.}
\end{frame}
\section{Results: phenotypic diversity and safe strategies}

\begin{frame}
  \begin{figure}
    \centering
    \frame{\includegraphics{../sims/asymmetric/plots/strat_phe_0_i/rates.pdf}}
  \end{figure}
\end{frame}

\begin{frame}
  \begin{figure}
    \centering
    \frame{\includegraphics{../sims/asymmetric/plots/strat_phe_0_i/avg_strat_phe_0.pdf}}
  \end{figure}
\end{frame}

\begin{frame}
  \begin{figure}
    \centering
    \frame{\includegraphics{../sims/asymmetric/plots/n_agents_i/avg_strat_phe_0.pdf}}
  \end{figure}
\end{frame}

\begin{frame}
  \begin{figure}
    \centering
    \frame{\includegraphics{../sims/asymmetric/plots/strat_phe_0_i/dist_phe_0.pdf}}
  \end{figure}
\end{frame}

\begin{frame}
  \begin{minipage}{0.3\textwidth}
    Some text...
    \vfill
  \end{minipage}
  \hfill
  \begin{minipage}{0.6\textwidth}
    \begin{figure}
      \frame{\includegraphics[scale=0.8]{../sims/extended/plots/strat_phe_0_i/avg_strat_phe_0.pdf}}
    \end{figure}
    \begin{figure}
      \frame{\includegraphics[scale=0.8]{../sims/extended/plots/strat_phe_0_i/dist_phe_0.pdf}}
    \end{figure}
  \end{minipage}
\end{frame}

\begin{frame}
  \begin{figure}
    \centering
    \frame{\includegraphics{../sims/extended/plots/strat_phe_0_i/dist_phe_0.pdf}}
  \end{figure}
\end{frame}

\section{Conclusions and future work}

\begin{frame}
  \begin{block}{Conclusions}
    \begin{itemize}
      \item ...
    \end{itemize}
  \end{block}
  \begin{minipage}{0.48\textwidth}
    \begin{block}{Future work}
      \begin{itemize}
        \item ...
      \end{itemize}
    \end{block}
    \vfill
  \end{minipage}
  \hfill
  \begin{minipage}{0.48\textwidth}
    \begin{figure}
      \centering
      \frame{\includegraphics[scale=0.6]{../sims/asymmetric/plots/strat_phe_0_i/fitness.pdf}}
    \end{figure}
  \end{minipage}
\end{frame}

\section*{Acknowledgments}

\begin{frame}
  \begin{center}
    \Large{Thank you for your attention.}
  \end{center}
\end{frame}

\appendix

\section*{Numerical methods}

\begin{frame}
  \begin{block}{Gillespie algorithm}
    \begin{equation*}
      \mathbf{X}(t) = (X_1(t),\dots,X_N(t))
    \end{equation*}
    \begin{equation*}
      a_j(\mathbf{X}) = c_j h_j(\mathbf{X}), \ a_0(\mathbf{X}) = \sum_{j=1}^M a_j(\mathbf{X})
    \end{equation*}
    \begin{equation*}
      \tau = \frac{1}{a_0(\mathbf{X})}\ln\!\left(\frac{1}{r_1}\right), \ r_1 \sim U(0,1)
    \end{equation*}
    \begin{equation*}
      \sum_{j=1}^{\mu-1} a_j(\mathbf{X})< r_2 a_0(\mathbf{X})\le \sum_{j=1}^{\mu} a_j(\mathbf{X}), \ r_2 \sim U(0,1)
    \end{equation*}
    \begin{equation*}
      \mathbf{X}(t+\tau) = \mathbf{X}(t) + \boldsymbol{\nu}_\mu
    \end{equation*}
  \end{block}

  \note[item]{Mention the care needed to analyze the results of this simulations.}
\end{frame}

\end{document}
