\documentclass{beamer}

\usepackage[T1]{fontenc}
\usepackage[utf8]{inputenc}

\usepackage{lmodern}

\usepackage{mathtools,amssymb}

\usepackage{graphicx}
\usepackage{xcolor}
\usepackage{tikz}

\usepackage{booktabs}

\definecolor{red_UCM}{HTML}{b01131}

\usetheme{Berlin}
\useoutertheme{infolines}
\useinnertheme{rectangles}
\usecolortheme[named=red_UCM]{structure}

\usetikzlibrary{calc, positioning}

\AtBeginSection[]
{
  \begin{frame}
    \tableofcontents[currentsection]
  \end{frame}
}

\setbeameroption{show notes}

\setbeamertemplate{note page}{\vskip 0.5em \insertnote}
\addtobeamertemplate{note page}{}{\thispdfpagelabel{notes:\insertframenumber}}

\setbeamertemplate{navigation symbols}{}

\usebackgroundtemplate{\tikz[overlay,remember picture]\node[opacity=0.03125]at (current page.center){\includegraphics[scale=0.5]{images/UCM-logo.png}};}

\title[XXII GISC Workshop]{Evolving Safely: Adaptation and Phenotypic Diversity under Environmental and Demographic Constraints}
\author[]
{
  \texorpdfstring{{\large Mendívil Carboni, Marco\inst{1}}\\\vspace{1ex} Mazo Torres, Juan José\inst{1} (dir.) \and Dinis Vizcaíno, Luis Ignacio\inst{1} (dir.)}{Marco Mendívil Carboni}
} 
\institute[Universidad Complutense de Madrid]
{
  \inst{1}
  Grupo Interdisciplinar de Sistemas Complejos (GISC) and Departamento de Estructura de la Materia, Física Térmica y Electrónica, Universidad Complutense de Madrid
}

\begin{document}

\frame{\titlepage}

\begin{frame}
  \tableofcontents
\end{frame}

\section{Introduction and objectives}

\begin{frame}{Motivation}
  \begin{minipage}{0.6\textwidth}
    \begin{itemize}
      \item Adaptation in fluctuating environments.
      \item Do populations tend to evolve toward the strategy that maximizes their mean growth rate $\langle\mu\rangle$?
    \end{itemize}
  \end{minipage}
  \hfill
  \begin{minipage}{0.3\textwidth}
    \begin{figure}
      \includegraphics[width=\textwidth]{images/changing-environment.jpg}
    \end{figure}
  \end{minipage}

  \note[item]{Adaptation in fluctuating environments is a central problem in evolutionary biology.}
  \note[item]{A classical view of evolution suggests that populations tend to evolve toward the strategy that maximizes their mean growth rate $\langle\mu\rangle$.}
\end{frame}

\section{The model: a Markov process}

\begin{frame}{The model}
  \begin{table}
    \centering
    \begin{tabular}{c c c}
      \toprule
      symbol & name                     & value \\
      \midrule
      $N_i$  & initial number of agents & $100$ \\
      \bottomrule
    \end{tabular}
  \end{table}
\end{frame}

\section{Results: phenotypic diversity and safe strategies}

\section{Conclusions and future work}

\begin{frame}{Conclusions}
  \begin{itemize}
    \item Item example.
  \end{itemize}
  \note[item]{Note example.}
\end{frame}

\section{Acknowledgments}

\begin{frame}
  \begin{center}
    \Large{Thank you for your attention.}
  \end{center}
\end{frame}

\section{Appendix}

\begin{frame}{Numerical methods: Gillespie algorithm}
\end{frame}

\end{document}
