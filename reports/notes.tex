\documentclass[a4paper,11pt]{article}

\usepackage[T1]{fontenc}
\usepackage[utf8]{inputenc}

\usepackage{lmodern}

\usepackage[margin=2.5cm]{geometry}
\usepackage{setspace}

\usepackage{mathtools,amssymb}

\usepackage{graphicx}
\usepackage{xcolor}
\usepackage{tikz}

\usepackage[hidelinks,pdfusetitle]{hyperref}

\setstretch{1.15}

\title{Notes on project mutare''}
\author{Marco Mendívil Carboni}

\begin{document}

\maketitle

\section{Introduction and objectives}

It is known...

\section{The model: a Markov process}

To study...

mutare simulates a stochastic agent-based model of adaptation in uncertain environments with the following characteristics:
\begin{itemize}
    \item The environment is a discrete variable with n env possible values and follows a Markov chain defined by the transition rates rates trans.
    \item Each agent carries a phenotype, a discrete variable with n phe possible values, and a phenotypic strategy, a distribution over phenotypes.
    \item Agents may duplicate or die according to environment and phenotype specific rates (rates birth and rates death).
    \item The offspring's phenotype is sampled from the parent's phenotypic strategy.
    \item The offspring inherits the parent's phenotypic strategy, but with probability prob mut it suffers a random mutation and changes completely.
    \item At every simulation step, the population is capped at its initial size (n agents) and reinitialized if extinction is reached.
    \item Initially, if strat phe is set, all agents will share that same strategy; otherwise, they will each have a random strategy.
\end{itemize}

During the simulation, every steps per save steps, the following observables are computed and saved:
\begin{itemize}
    \item Current simulation time
    \item Time until the next event
    \item Instantaneous population growth rate
    \item Number of extinctions so far
    \item Average phenotypic strategy
    \item Standard deviation of the phenotypic strategy
\end{itemize}

Every steps per file steps, the simulation is stopped and a new output file is written to disk.


\begin{equation}
    ...
\end{equation}

\begin{figure}[p]
    \centering
    \includegraphics{../plots/simulations/default/extinct_rate.pdf}
    \caption{Extinction rate as a function of growth rate for the default parameters.}
    \label{fig:extinct_rate_default}
\end{figure}

\begin{figure}[p]
    \centering
    \includegraphics{../plots/simulations/default/growth_rate.pdf}
    \caption{Growth rate as a function of the initial strategy for the default parameters.}
    \label{fig:growth_rate_default}
\end{figure}

\begin{figure}[p]
    \centering
    \includegraphics{../plots/simulations/default/avg_strat_phe.pdf}
    \caption{Average strategy as a function of the initial strategy for the default parameters.}
    \label{fig:avg_strat_phe_default}
\end{figure}

\begin{figure}[p]
    \centering
    \includegraphics{../plots/simulations/default/dist_strat_phe.pdf}
    \caption{Strategy distribution as a function of the initial strategy for the default parameters.}
    \label{fig:dist_strat_phe_default}
\end{figure}

\begin{figure}[p]
    \centering
    \includegraphics{../plots/simulations/biological/extinct_rate.pdf}
    \caption{Extinction rate as a function of growth rate for the biological parameters.}
    \label{fig:extinct_rate_biological}
\end{figure}

\begin{figure}[p]
    \centering
    \includegraphics{../plots/simulations/biological/growth_rate.pdf}
    \caption{Growth rate as a function of the initial strategy for the biological parameters.}
    \label{fig:growth_rate_biological}
\end{figure}

\begin{figure}[p]
    \centering
    \includegraphics{../plots/simulations/biological/avg_strat_phe.pdf}
    \caption{Average strategy as a function of the initial strategy for the biological parameters.}
    \label{fig:avg_strat_phe_biological}
\end{figure}

\begin{figure}[p]
    \centering
    \includegraphics{../plots/simulations/biological/dist_strat_phe.pdf}
    \caption{Strategy distribution as a function of the initial strategy for the biological parameters.}
    \label{fig:dist_strat_phe_biological}
\end{figure}

\end{document}
